\documentclass[a4paper,11pt]{article}
\usepackage{cmap}
\usepackage{polski}
\usepackage[T1]{fontenc}
\usepackage[utf8]{inputenc}
\usepackage{graphicx}
\usepackage{minted}
\usepackage{amsmath}
\newcommand\Inn{%
  \mathrel{\ooalign{$\subset$\cr\hfil\scalebox{0.8}[1]{$=$}\hfil\cr}}%
}

\title{Opis Algorytmów MPI Wyznaczania Liczb Pierwszych}
\author{Artur Frącala
        Dawid Kurzydło}
\date{kwiecień 2019}

\begin{document}
  \begin{center}\Large
Aplikacje Internetowe i Rozproszone
\end{center}
  \hrule
  {\let\newpage\relax\maketitle}
  \hrule


  \section{Algorytm I}
    Przedstawimy program wyznaczania liczb pierwszych z przedziału 2 .. n, dla pewnej liczby naturalnej n.
    Jest on zbudowany na podstawie dekompozycji danych.
    Korzysta się w nim z obserwacji, iż do wyznaczania liczb pierwszych z przedziału B = $\lfloor\sqrt{n}\rfloor$ + 1..n wystarczy znajomość liczb pierwszych z przedziału A = 2 .. $\lfloor\sqrt{n}\rfloor$.
    Każda liczba złożona należąca do przedziału B dzieli się bowiem przez jedną lub więcej liczb z przedziału A.
    Nazwijmy liczby pierwsze z przedziału A podzielnikami.
    Wówczas, aby sprawdzić czy dowolna liczba j należąca do B jest złożona, wystarczy sprawdzić, że dzieli się ona bez reszty przez któryś z podzielników.
    Przykładowo, jeśli chcemy znaleźć liczby pierwsze z przedziału 2 .. 10000, to najpierw znajdujemy liczby pierwsze z przedziału 2 .. 100 stosując np. sito Eratostenesa.
    Podzielników takich jest 25, są nimi liczby: 2, 3, 5, . . . , 97.
    Następnie dla dowolnej liczby j należącej do B należy sprawdzić, czy dzieli się ona przez któryś z podzielników.
    Aby zminimalizować koszty komunikacji, w pierwszej fazie programu wszystkie procesy wyznaczają podzielniki z przedziału A.
    W fazie drugiej następuje dekompozycja danych, tj. podział zakresu liczbowego B = $\lfloor\sqrt{n}\rfloor$ + 1 .. n na p pod przedziałów o długości d = $\lceil$(n - $\lfloor\sqrt{n}\rfloor$)/p$\rceil$, gdzie p jest liczbą procesów realizujących obliczenia.
    Procesom o numerach 0, 1, ..., p - 1 przydzielane są odpowiednio pod przedziały S + 1 .. S + d, S + 1 + d .. S + 2d, ..., S + 1 + (p - 1)d .. n, gdzie S =$\lfloor\sqrt{n}\rfloor$.
    W przydzielonych pod przedziałach procesy pracując równolegle wyznaczają liczby pierwsze przez eliminację liczb złożonych będących wielokrotnościami podzielników z przedziału A.

\end{document}
